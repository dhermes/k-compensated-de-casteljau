\documentclass[letterpaper,10pt]{article}

\usepackage[margin=1in]{geometry}

\usepackage{amsthm,amssymb,amsmath}
%% H/T: https://tex.stackexchange.com/a/202168/32270
\usepackage{textcomp}

\usepackage[usenames, dvipsnames]{color}
\usepackage{hyperref}
\hypersetup{
  colorlinks=true,
  urlcolor=blue,
  citecolor=ForestGreen,
  pdfinfo={
    CreationDate={D:20180425160342},
    ModDate={D:20180425160342},
  },
}

\usepackage{embedfile}
\embedfile{\jobname.tex}

\usepackage{fancyhdr}
\pagestyle{fancy}
\lhead{\(K\)-compensated de Casteljau}
\rhead{Danny Hermes}

\renewcommand{\headrulewidth}{0pt}
\newcommand{\fl}{\operatorname{fl}}
\newcommand{\eps}{\varepsilon}

\begin{document}

\tableofcontents

\section{de Casteljau's Method}

Consider de Casteljau's method to evaluate a degree \(n\)
polynomial in Bernstein-B\'{e}zier form with control points \(p_j\):
\begin{align*}
    b_j^{(0)} &= p_j \\
    b_j^{(k)} &= (1 - s) b_j^{(k - 1)} + s b_{j + 1}^{(k - 1)} \\
    b(s) &= b_0^{(n)}.
\end{align*}

\subsection{Selection of Test Cases}

From \cite{Delgado2015}

\begin{quote}
  We can observe that, in this case, the algorithm with a good
  behavior everywhere is the de Casteljau algorithm
\end{quote}

\noindent In the same paper (when referring to \cite{Bezerra2013}):

\begin{quote}
  assuming that all control points are positive. This assumption avoided
  ill-conditioned polynomials. In this section, we shall show that this is
  a natural assumption in Computer Aided Geometric Design (from now on,
  C.A.G.D.) and that it permits to assure high relative precision for the
  evaluation through a large family of representations in C.A.G.D.
\end{quote}

\noindent From the same author, in \cite{Mainar2005}:

\begin{quote}
  Let us observe that in this case, the de Casteljau algorithm presents
  better stability properties for the evaluation near the roots. In fact,
  the de Casteljau algorithm has good behaviour even when using simple
  precision, although the running error bound is not so accurate in points
  close to the roots.
\end{quote}

\subsection{\texorpdfstring{\(K\)}{K}-Fold Error Filtering}

Empirically, it seems the process takes
\[(15K^2 - 34K + 26)T_n + K + 5\]
flops to evaluate a degree \(n\) polynomial. (Here \(T_n\) is the
\(n\)th triangular number.)

\section{Bogus Section for Refs}

Here they are, for now
\begin{itemize}
  \item Compensated Horner (\(K = 2\)) (\cite{langlois_et_al:DSP:2006:442})
  \item Compensated de Casteljau (\cite{Jiang2010})
  \item Newton with compensated Horner (\cite{Graillat2008})
  \item \(K\)-fold Sum (\cite{Ogita2005})
  \item \(K\)-fold Horner (\cite{Graillat2009})
\end{itemize}

\bibliography{paper}
\bibliographystyle{alpha}

\end{document}
