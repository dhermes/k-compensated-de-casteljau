%%%%%%%%%%%%%%%%%%%%%%%%%%%%%%%%%%%%%%%%%
% Professional Formal Letter
% LaTeX Template
% Version 1.0 (28/12/13)
%
% This template has been downloaded from:
% http://www.LaTeXTemplates.com
%
% Original author:
% Brian Moses (http://www.ms.uky.edu/~math/Resources/Templates/LaTeX/)
% with extensive modifications by Vel (vel@latextemplates.com)
%
% License:
% CC BY-NC-SA 3.0 (http://creativecommons.org/licenses/by-nc-sa/3.0/)
%
%%%%%%%%%%%%%%%%%%%%%%%%%%%%%%%%%%%%%%%%%

\documentclass[11pt,a4paper]{letter}

\usepackage{graphicx}
\usepackage{hyperref}
\hypersetup{
  colorlinks=true,
  pdfinfo={
    CreationDate={D:20180516090844},
    ModDate={D:20180516090844},
  },
}

% Create a new command for the horizontal rule in the document
% which allows thickness specification
\makeatletter
\def\vhrulefill#1{\leavevmode\leaders\hrule\@height#1\hfill \kern\z@}
\makeatother

%------------------
% DOCUMENT MARGINS
%------------------

\textwidth 6.75in
\textheight 9.25in
\oddsidemargin -.25in
\evensidemargin -.25in
\topmargin -1in
\longindentation 0.50\textwidth
\parindent 0.4in

%--------------------
% SENDER INFORMATION
%--------------------

\def\Who{Danny Hermes}
\def\School{UC Berkeley}
\def\Where{Department of Mathematics}
\def\Address{970 Evans Hall \#3840}
\def\CityZip{Berkeley CA 94720-3840}
\def\Email{dhermes@berkeley.edu}

%-----------------------------------
% HEADER AND FROM ADDRESS STRUCTURE
%-----------------------------------

\address{
\includegraphics[width=1in]{uc_berkeley_seal.pdf} % Include the logo of your institution
\hspace{5.1in} % Position of the institution logo, increase to move left, decrease to move right
\vskip -1.07in~\\ % Position of the text in relation to the institution logo, increase to move down, decrease to move up
\Large\hspace{1.5in}UNIVERSITY \hfill ~\\[0.05in] % First line of institution name, adjust hspace if your logo is wide
\hspace{1.5in}OF CALIFORNIA \hfill \normalsize % Second line of institution name, adjust hspace if your logo is wide
\makebox[0ex][r]{\bf \Who}\hspace{0.08in} % Print your name and title with a little whitespace to the right
~\\[-0.11in] % Reduce the whitespace above the horizontal rule
\hspace{1.5in}\vhrulefill{1pt} \\ % Horizontal rule, adjust hspace if your logo is wide and \vhrulefill for the thickness of the rule
\hspace{\fill}\parbox[t]{2.85in}{ % Create a box for your details underneath the horizontal rule on the right
\footnotesize % Use a smaller font size for the details
\textit{\School}  \\
\textit{\Where}   \\
\textit{\Address} \\
\textit{\CityZip} \\
\textit{\Email}   \\
}
\hspace{-1.4in} % Horizontal position of this block, increase to move left, decrease to move right
\vspace{-1in} % Move the letter content up for a more compact look
}

%----------------------
% TO ADDRESS STRUCTURE
%----------------------

\def\opening#1{\thispagestyle{empty}
{\centering\fromaddress \vspace{0.6in} \\ % Print the header and from address here, add whitespace to move date down
\hspace*{\longindentation}\today\hspace*{\fill}\par} % Print today's date, remove \today to not display it
{\raggedright \toname \\ \toaddress \par} % Print the to name and address
\vspace{0.4in} % White space after the to address
\noindent #1 % Print the opening line
}

\signature{
  \includegraphics[width=2.5in]{daniel_hermes_signature.jpg}
}

\long\def\closing#1{
\vspace{0.1in} % Some whitespace after the letter content and before the signature
\noindent % Stop paragraph indentation
\hspace*{\longindentation} % Move the signature right
\parbox{\indentedwidth}{\raggedright
#1 % Print the signature text
\vskip 0.05in % Whitespace between the signature text and your name
\fromsig}} % Print your name and title

\begin{document}

\begin{letter}
{Editors \\
Applied Mathematics and Computation}

\opening{Dear Editors,}

I am pleased to submit the original research article
``Compensated de Casteljau algorithm in \(K\) times the working precision''
for consideration for publication in Applied Mathematics and Computation.
This manuscript builds on prior work of Jiang et al in evaluating a polynomial
in Bernstein form with greater accuracy. It follows Graillat et al in
producing a method for evaluation that produces results as accurate as if
performed in \(K\) times the working precision, then rounded back into the
working precision.

In this manuscript, error-free transformations for the sum and product
are used to track the \textbf{exact} round-off error throughout the
de Casteljau algorithm. At each stage of computation, round-off error is
passed on to first order errors, then to second order errors, and so on.
After the computation has been ``filtered'' \((K - 1)\) times via this
process, the resulting output is as accurate as the de Casteljau algorithm
performed in \(K\) times the working precision.

I believe that this manuscript is appropriate for publication by
Applied Mathematics and Computation because it describes a computational
method for extending the accuracy of a well-studied algorithm to
arbitrarily high precision. The accompanying error analysis confirms
the numerical results.

I confirm that this manuscript has not been published elsewhere and is not
under consideration by another journal. All authors have approved the
manuscript and agree with its submission to Applied Mathematics and
Computation.

Thank you for your consideration!

\closing{Sincerely,}

\end{letter}
\end{document}
